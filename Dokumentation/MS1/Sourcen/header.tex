
\documentclass[a4paper,10pt,fleqn]{article} % Definiert Papier = A4;
                                            % Schriftgrösse = 10Punkte;
                                            % Mathe.-Gl. Modus = linksbündig
                                            % (siehe http://lefti.amigager.de/latex/Aufbau.html)

\usepackage[utf8x]{inputenc}                % utf8x kann alle Textcodierungen interpretieren
\usepackage[T1]{fontenc} %Schriftcodierung mit UTF-8
\usepackage{textcomp} %Erweiterung von fontenc
\usepackage{lmodern} %Erweiterung des Zeichensatzes

\usepackage{graphics}                       % Package für Einfügen/Anpassen von Grafiken
\usepackage{graphicx}
\usepackage{wrapfig}                        % Package zum Einfügen von Textumflossenen Bilder

\usepackage[english,ngerman]{babel}         % ngerman = Neues Deutsch; babel = internationalisierung einschalten
\usepackage[babel,german=quotes]{csquotes}  % Deutsche Gänsefüssechen

\usepackage{hyperref}

\usepackage{amsmath}                        
\usepackage[all]{xy}
%\usepackage[xindy]{glossaries}
\usepackage{makeidx}
\usepackage{pdfpages}
\usepackage{graphicx}
\usepackage{printlen}

\usepackage{blindtext}
\usepackage{lipsum}

\usepackage{multicol}

\usepackage{multirow}

\usepackage{verbatim}

\usepackage{color}

\usepackage{eurosym}

%\usepackage{hyperref}
\usepackage{acronym}

\usepackage{enumitem}

\usepackage{setspace}

\usepackage{threeparttable} %benötigt um Fussnoten in einer Tabelle zu machen

\usepackage{spreadtab} %benötigt um in Tabellen zu rechnen
\usepackage{numprint}
%
% Rechnen in Tabellen. Zeichen für den Dezimalpunkt
%
\STsetdecimalsep{{.}}

\definecolor{darkgreen}{rgb}{0,0.6,0}
\usepackage{listings}
\lstset{language=[LaTeX]TeX}
\lstloadlanguages{TeX}
\lstset{basicstyle=\ttfamily\footnotesize,
        numbers=left,
        numberstyle=\tiny,
        numbersep=5pt,
        breaklines=true,
        texcsstyle=\color{black},
        backgroundcolor=\color{gray!10},
        %commentstyle=\color{darkgreen},
        %keywordstyle=\color{red}\bfseries,
        %stringstyle=\color{blue}\bfseries,
        frame=single,
        tabsize=2,
        rulecolor=\color{black!30},
        title=\lstname,
        escapeinside={\%*}{*)},
        breaklines=true,
        breakatwhitespace=true,
        framextopmargin=2pt,
        framexbottommargin=2pt,
        inputencoding=utf8x,
        extendedchars=true,
        literate={Ö}{{\"O}}1
                 {Ä}{{\"A}}1
                 {Ü}{{\"U}}1
                 {ü}{{\"u}}1
                 {ä}{{\"a}}1
                 {ö}{{\"o}}1 
       }
%
% hypersetup etwas genauer
%
\hypersetup{pdftex=true,
            hyperfigures=true,
            hyperindex=true,
            bookmarks=true,
            bookmarksopen=true,
            bookmarksnumbered=true,
            %pdfborder={0 0 0},
            hypertexnames=true,
            colorlinks=true,
            pagebackref=false,
            linktocpage=false,% link "`behing"'
            plainpages=false,
            linkcolor=black,%blue,
            %anchorcolor=black,% Color for anchor text.
            citecolor=black,%green,%Color for bibliographical citations in text.
            filecolor=black,%magenta,%Color for URLs which open local files.
            menucolor=black,%red,%Color for Acrobat menu items. pagecolor color red Color for links to other pages, but currently unused
            urlcolor=black,%red,
            pdfstartview=Fit,
            pdfview={XYZ null null null},
            pdfpagelabels,
            pageanchor=true,
            hypertexnames=false
           }
%
% hypersetup etwas rudimentärer
%
%{
    %colorlinks,
    %citecolor=black,
    %filecolor=black,
    %linkcolor=black,
    %urlcolor=black
%}

\usepackage{url}

\usepackage{cite}
\usepackage{apacite}

\usepackage{pdfpages}                        % Packet für PDF Dateimanipulation laden

\usepackage{fancyhdr}                        % http://en.wikibooks.org/wiki/LaTeX/Page_Layout#Customising_with_fancyhdr

\usepackage{siunitx}                         % benötigt um Tabellen nach einem . auszurichten


\bibliographystyle{apacite}


\pagestyle{fancy} %eigener Seitenstil
\fancyhf{} %alle Kopf- und Fusszeilenfelder bereinigen

\addtolength{\textwidth}{1.5cm}
\addtolength{\evensidemargin}{-5mm}
\addtolength{\oddsidemargin}{-5mm}

\addtolength{\headwidth}{1.5cm}

%\rhead{\setlength{\unitlength}{1mm}
%\begin{picture}(-2,7)
%    \includegraphics[width=35mm]{hslu_logo2.PNG}
%\end{picture}}

\fancyhead[L]{Projektarbeit: Autonomer Ballwerfer} %Kopfzeile links
\fancyhead[C]{} %zentrierte Kopfzeile
\fancyhead[R]{HSLU - T\&A}
%\fancyhead[R]{\includegraphics[scale=0.25]{hslu_logo2.PNG}}
\renewcommand{\headrulewidth}{0.4pt} % obere Trennlinie
\fancyfoot[L]{PREN Team 32}
\fancyfoot[C]{HS - 2014}
\fancyfoot[R]{\thepage}
\renewcommand{\footrulewidth}{0.4pt} % untere Trennlinie

%
% Anpassung der Darstellung des Literaturverzeichnis
%
\let\oldbibliography\thebibliography
\renewcommand{\thebibliography}[1]{%
  \oldbibliography{#1}%
  \setlength{\itemsep}{10pt}%
}
%
% Abbildungen und Tabellen nur mit Kurzform angeben
%
\addto\captionsngerman{
\renewcommand{\figurename}{Abb.}
\renewcommand{\tablename}{Tab.}
}
%
% Benötigt um von zwei Stellen im Text auf dieselbe Fusszeile zu referenzieren
%
\newcommand{\footnoteremember}[2]{%
  \footnote{#2\label{#1}}
  \newcounter{#1}
  \setcounter{#1}{\value{footnote}}
}
\newcommand{\footnoterecall}[1]{%
  \hyperref[#1]{\footnotemark[\value{#1}]}
} 
%
%\makeglossaries
%\include{wa/wa_glossar}
%