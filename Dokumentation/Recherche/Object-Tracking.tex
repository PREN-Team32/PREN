<<<<<<< HEAD
\section{Object-Tracking – Objekt Verfolgung}
=======
\section{Object-Tracking – Objekt Verfolgung}
Es wurden die verschiedenen Möglichkeiten angeschaut wie ein Objekt erkannt werden kann. Die Möglichkeiten sind mit den gestellten Anforderungen und auf ihre Realisierbarkeit verglichen worden. Genauigkeit und Geschwindigkeit spielen eine gewichtige Rolle wie auch der benötigte Aufwand und Kosten für die Umsetzung.

>>>>>>> origin/master
\subsection{Google Obj-Tracking with OpenCV}
Zur Objekterkennung läuft auf dem Android Smartphone eine App, welche mithilfe der Kamera die Objekterkennung durchführt und die Informationen an den Controller weitergibt. Es ist eine Anleitung und Source Code vorhanden. Als Framework wird OpenCV verwendet. Google Obj-Tracking ist eine Bibliothek die eine Vielzahl von Bildverarbeitungsalgorithmen bereitstellt. Das Framework ist sehr gut beschrieben, es sind viele Tutorials vorhanden.

\subsection{Accord.Net}
Accord.Net ist eine OpenSource Bibliothek für das .Net Framework. Es werden Code Beispiele und Tutorials angeboten. Gut Dokumentiert.

\subsection{Ultrasonic / Ultraschall }
Beschreibt wie die Erkennung von Objekten mithilfe von Ultraschallsensoren. Ultraschallsensoren können sehr günstig sein allerdings ist die Genauigkeit je nach anwendungsfall nicht sehr gross. Die Temperatur beeinflusst die Genauigkeit massgeblich. 

\subsection{Infrarot}
Infrarot Sensoren geben einen Infrarot Lichtstrahl ab, ein Sensor erkennt dann die Rückstrahlung womit sich Objekte erkennen lassen. Die Distanz beträgt je nach Sensor zwischen 1 – 150 cm. Infrarot kann durch äussere Einflüsse, Lichtquellen, an Genauigkeit einbüsst.

\subsection{Laser-Scanning}
Die Reichweite eines Lasers beträgt je nach Art bis zu mehreren hundert Metern. Die Genauigkeit liegt je nach Auswertungshardware im Milimeterbereich. Die Kosten für Laser-Systeme sind allerdings sehr hoch, wie auch deren Gewicht.

