\subsection{Fahrantrieb}
Bei der Recherche des Fahrantriebs wurde hauptsächlich darauf geachtet, dass ein guter Stand des Produkts gewährleistet ist. Zum einen bietet sich hier der Raupenantrieb an. Diese Technologie hat eine grosse Kontaktfläche mit dem Boden und ist sehr manövrierfähig. Das Laufwerk kann je nach Grösse und Anforderungen spezifisch ausgelegt werden. Weiter gibt es einen Schraubenantrieb. Hier sind an der Unterseite des Produkts zwei gegenläufige Schrauben angebracht. Durch Bodenkontakt auf der gesamten Länge ist gute Stabilität gewährleistet.  Ausserdem kann sich das Produkt, als Eigenheit des Schraubenantriebes, vom Punkt aus gleichermassen vor und zurück, wie auch seitwärts bewegen. Als Nachteil ist hier die schlechte Traktion auf festem Untergrund. Das Luftkissenfahrzeug schwebt dank eines Überdruckes unter dem Fahrzeug wenige Zentimeter über dem Boden. Gelenkt wird mittels eines Propellers auf dem Fahrzeug, dessen Luftstrom gelenkt wird. Zuletzt ein konventioneller Antrieb via Reifen. Hier gibt es unzählige Ausführungsmöglichkeiten von Antrieb und Lenkung. 
