\subsection{Flugobjekte}
Als Flugobjekte wurden drei verschiedene Möglichkeiten ins Auge gefasst. Dazu zählt ein Quadcopter, eine Zeppelin und eine Rakete. Die Hauptschwierigkeit besteht bei der Steuerung der Objekte in der Flugphase. Eine weitere Teilschwierigkeit ist, eine berechenbare Flugbahn zu erreichen. 

\subsubsection{Quadrocopter}
\textbf{Vorteile}\\
Ein Quadrocopter kann als fertiger Baukasten gekaut und zusammengebaut werden. Die Flugsteuerung erfolgt über eine fertige Software.\\
\\
\textbf{Nachteile}\\
Die Steuerung des Quadrocopter ist sehr schwierig. Die Orientierung im Raum ist mit einer einfachen Software nicht möglich. Um eine bestimmte Flugbahn einzuhalten, brauchte man diverse Kameras, welche im Raum verteilt sind. Der ganze Quadrocopter und die Steuerung sind sehr kostenintensiv.\\
\\
\textbf{Umsetzbarkeit}\\
Eine Möglichkeit, für eine effiziente Umsetzung eines Quadrocopters in das Konzept ist fast 
Unmöglich. Die Kosten werden bei weitem überschritten. Die genaue Steuerung im Raum ist extrem schwierig. 

\subsubsection{Zeppelin}
\textbf{Vorteile}\\
Der Zeppelin kann als fertiger Baukasten gekauft werden. Die Modelle können der jeweiligen Hebelast angepasst werden. \\
\\
\textbf{Nachteile}\\
Der Auftriebskörper für eine kleine Masse zu heben, ist sehr gross. Die Steuerung des ganzen Zeppelins verläuft sehr träge.\\
\\
\textbf{Umsetzbarkeit}\\
Aus Platzgründen, welcher der Auftriebkörper benötigt, ist der Zeppelin sehr schwierig zu realisieren. \\
\\
\subsubsection{Rakete}
\textbf{Vorteile}\\
Sehr schneller Vortrieb des Wurfskörpers.\\
\\
\textbf{Nachteile}\\
Die Wurfbahn einer Rakete auf kleine Distanz ist fast unmöglich. \\
\\
\textbf{Umsetzbarkeit}\\
Die Umsetzung eines Raketenantriebes ist unmöglich.\\
\\
\\
Als erstes wurden diverse ins Auge gefasst, wie die Bälle durch die Luft befördert werden können. Dazu gibt es schon diverse fertige Lösungen, welche mit einigen Änderungen übernommen werden können. Die Webseite für den Bau eines Quadrocopters ist sehr ausführlich und genau beschrieben. Die Umsetzung ist jedoch mit viel Aufwand verbunden. Eine Alternative zum Quadrocopter bietet ein Zeppelin. Auch hier konnte im Internet bereits eine ausführliche Anleitung gefunden werden.
