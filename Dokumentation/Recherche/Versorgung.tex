\subsection{Versorgung}
Eine Möglichkeit um das Produkt mit Energie zu versorgen, ist ein Akkumulator. Es gibt verschiedene Typen: Blei-Akkus, Li-Ionen-Akku, Nickel-Cadmium-Akku (NiCd), Nickel-Metallhydird-Akku (NiMh). Jeder Typ hat verschiedene Vor- und Nachteile, die in der Tabelle \ref{tab:UebersichtVorNachTeil} ersichtlich sind.\\
\\
Ein grosser Vorteil besteht darin, dass ein Akkumulator nicht Teil des Produktegewichts ist. Wichtig für die anschliessende Auswahl des Akkumulators sind die Spannung, Strom, Kapazität des Akkumulators. In dieser Technologierecherche beschränkt man sich auf Eckdaten, die die Akkus auszeichnen, wie in der Tabelle \ref{tab:UebersichtAkku} ersichtlich.\\ 

\begin{table}[h!]
	\begin{tabular}{|p{1.5cm}|p{2.3cm}|p{3cm}|p{3cm}|} \hline
		          &\textbf{Energiedichte ($\frac{Wh}{kg}$})  & \textbf{Wirkungsgrad} & \textbf{Memory-Effekt}\\ \hline
		NiCd      & 40-60                                    & 70                    & Ja \\ \hline
		NiMH      & 70-90                                    & 70                    & Nein  \\ \hline
		Li-Ion    & 120-210                                  & 90                    & Nein \\ \hline
		Blei (Pb) & 30                                       & 60-70                 & Nein \\ \hline
	\end{tabular}
	\centering
	\caption{Übersicht der Akkumulatoren}
	\label{tab:UebersichtAkku} 
\end{table}

\begin{table}[h!]
	\begin{tabular}{|p{1.2cm}|p{5.3cm}|p{5.3cm}|} \hline
		          &\textbf{Vorteil}  & \textbf{Nachteil}\\ \hline
		NiCd      & \begin{itemize} \item Lange Lebensdauer \item Wartungsfreie Bauform \end{itemize} & \begin{itemize} \item In der EU verboten! \item Memory-Effekt (-> Kapazitätsverlust) \item Bei Defekt, sehr umweltschädlich \end{itemize} \\ \hline
		NiMH      & \begin{itemize} \item Hohe Kapazität \item Geeignet für Hochstromanwendungen \end{itemize} & \begin{itemize} \item Geringes Gewicht (kein Ballast) \item Hohe Selbstentladung 15\% pro Monat \end{itemize}   \\ \hline
		Li-Ion    & \begin{itemize} \item 5 Jahre funktionstüchtig \item Hohe Energiedichte \item Selbstentladung 1\% pro Monat \end{itemize} & \begin{itemize} \item Empfindlich auf falsche Behandlung \item Unter 1.5V kommt es zu Brandgefahr \item Geringes Gewicht (kein Ballast) \end{itemize} \\ \hline
		Blei (Pb) & \begin{itemize} \item 6 Jahre funktionstüchtig \item Hohe Strombelastbarkeit \item Hohes Gewicht (als Ballast) \end{itemize} & \begin{itemize} \item Selbstentladung 1\% pro Tag \item Nicht für mobilen Einsatz geeignet \end{itemize} \\ \hline
	\end{tabular}
	\centering
	\caption{Übersicht Vor- Nachteile der Akkumulatoren}
	\label{tab:UebersichtVorNachTeil} 
\end{table}

\subsubsection{Externe Versorgung}
In diesem Anschnitt ist die Versorgung mit Energie via Netzteil gemeint. Ein Vorteil eines Netzteils ist die stabile Energie- / Stromversorgung. Da eine Zuführung von einer Steckdose zum Spielfeld gewährleistet sein wird, fällt dies als Nachteil weg. Als Nachteil kann man jedoch auflisten, dass man das Netzteil nicht als Ballast verwenden kann. 

\subsubsection{Pneumatik}
Eine Versorgung mit Druckluft ist aufwendig und muss beim Spielfeld zur Verfügung gestellt werden. Ansonsten müsste man einen Kompressor mit Wartungseinheit organisieren. Zudem sind die Komponenten (Zylinder, Ventile, …) im Neuzustand seht teuer im Einkauf. 

\subsubsection{Hydraulik}
Die Versorgung mit Hydraulik-öl ist noch aufwendiger als jene mit Druckluft. Es muss ein eigenes System mit Pumpe, Schläuchen, Hydrauliköl und teuren Komponenten erstellt werden. Zudem entsteht bei einem Defekt resp. Unfall mit Hydrauliköl schnell ein grosser Sachschaden und erfordert einen grossen Reinigungsaufwand. 

