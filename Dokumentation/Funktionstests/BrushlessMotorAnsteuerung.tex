\subsection{Brushless Motoransteuerung}

\begin{tabular}{p{3.6cm}p{9.4cm}}
\textit{Typ}              & Ballmaschine\\ 
\textit{Datum}:           & 22.11.2014\\
\textit{Ort}:             & Labor HSLU\\
\textit{Tester}:          & Yves Studer\\
\textit{Ziel des Testes}: & Testen ob die Ansteuerung eines Brushless Motors in verhältnismässiger Zeit realisierbar ist. Weiter wird versucht, die Signale von Hallsensoren aus der Ansteuerung zu rekonstruieren, da die Steuerung des Motors mittels Hall-Sensoren einfacher realisiert werden kann.\\
\textit{Fazit / Verbesserungs-\newline vorschlag}: & Die Ansteuerung des Motors ist mit erstaunlich wenig Aufwand möglich. Den Motor drehen lassen ist kein Problem. Ob das maximale Drehmoment erreicht werden kann, bedarf weiteren Abklärungen. Im Versuchsaufbau konnte mit einer Wirbelstrombremse rund 100W elektrischer Leistung im Motor umgesetzt werden. \\ 
\textit{Ziel erreicht}:& Ja\\
\end{tabular}
 