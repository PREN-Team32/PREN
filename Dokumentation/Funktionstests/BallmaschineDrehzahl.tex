\subsection{Ballmaschine (Eruieren der Nenndrehzahl)}

\begin{tabular}{p{3.6cm}p{\textwidth-3.6cm-0.7cm}}
\rule{0pt}{11pt}\textit{Typ}              & Ballmaschine \\ 
\rule{0pt}{11pt}\textit{Datum}:           & 06.11.2014   \\
\rule{0pt}{11pt}\textit{Ort}:             & Labor HSLU \\
\rule{0pt}{11pt}\textit{Tester}:          & Matteo, Yves, Pascal\\
\rule{0pt}{11pt}\textit{Ziel des Testes}: & Bestimmung der ungefähren Nenndrehzahl der 
Schwungräder sowie die Optimierung des Wurfwinkel.  \\
\rule{0pt}{11pt}\textit{Aufbau / Ablauf}: & In diesem Test wurden zwei DC-Motoren eingesetzt, 
die jeweils von einem  Rohde \& Schwarz Labornetzgerät gespiesen wurde. Die Drehzahl der Motoren 
konnte mittels der Spannung variiert werden. Die Messung der Drehzahl erfolgte über ein berührendes 
Drehzahlmessgerät, das aus dem Physikbestand der Hochschule Luzern ausgeliehen wurde.\\
\rule{0pt}{11pt}\textit{Fazit / Verbesserungs-\newline vorschlag}: & Die Ballzuführung 
muss automatisiert und gleichbleibend sein, damit genaue Aussagen über die Drehzahl und dadurch die Wurfweite 
gemacht werden können. Weiter ist festgestellt worden, dass es eine markanten Unterschiedliche zwischen unterschiedlichen Tennisballmarken gibt. Diese varieren im Durchmesser und der Härte. Dadurch variert auch wieder die Wurfweite. Für die nächsten Test, müssen fünf für den Wettkampf zugelassene Bälle verwendet werden. Die Drehzahl der Schwungräder bei einem Abstand von 1.8 m beträgt ungefähr 6000 U/min. Diese Zahl wird noch varieren, wenn andere Schwungräder eingesetzt werden. Natürlich ist dies auch von der Zuführgeschwindigkeit abhängig. Für eine Auswahl von einem geeigneten Brushless-Motor, reicht die ungefähre Angabe der Drehzahl aus.
\end{tabular}
\begin{figure}[h!]
	\includegraphics[width=0.7\textwidth,clip,trim=0mm 10cm 0mm 12cm]
	{Funktionstests/Bilder/Ballmaschine_Drehzahl1.jpg}
	\centering
	\caption{Funktionsmuster Ballmaschine} 
\label{abb:Ballmaschine_Drehzahl}
\end{figure}