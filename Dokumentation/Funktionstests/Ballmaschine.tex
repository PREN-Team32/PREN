\subsection{Ballmaschine}
\begin{tabular}{p{3.6cm}p{9.4cm}}
\rule{0pt}{11pt}\textit{Typ}              & Ballmaschine \\ 
\rule{0pt}{11pt}\textit{Datum}:           & 18.10.2014   \\
\rule{0pt}{11pt}\textit{Ort}:             & Labor HSLU \\
\rule{0pt}{11pt}\textit{Tester}:          & Gruppe 32 \\
\rule{0pt}{11pt}\textit{Ziel des Testes}: & Das Ziel dieses Testes bestand darin, den gebauten Prototyp (Ballmaschine) auf die Genauigkeit und Wurfweite zu testen, weitere Erkenntnisse über die Drehzahl der Räder zu eruieren und die erforderliche Stromstärke unter realen Bedingungen testen.  \\
\rule{0pt}{11pt}\textit{Fazit / Verbesserungs-\newline vorschlag}: & Die Wurfmaschine kann mit einigen Verbesserungen sehr gute und genaue „Schüsse“ erzielen. Zu verbessern sind:
\begin{itemize}
    \item Stabilere Achsen
    \item genauere und gleichmässige Zuführung der Bälle.
    \item einstellbares Grundgerüst
\end{itemize}\\
\end{tabular}

Der Wurfmechanismus besteht aus zwei Kunststoffrädern mit einer weichen Pneubeschichtung. Die Räder sind an einer Metallplatte, direkt auf der Welle der Motoren angebracht. Die Motoren sind zwei Gleichstrommotoren, welche direkt an ein Netzteil angeschlossen sind. Die Drehzahl wird über die Spannungsstärke geregelt. Der Abwurfwinkel ist durch die Stellung der Metallplatte gegeben, welche in einem beweglichen Schraubstock eingespannt ist. Dadurch können diverese Winkel eingestellt werden. Die Zuführung der Bälle erfolgt manuell über eine Schiene.
