\subsubsection{Controller}
\label{sec:Controller}
	\begin{wrapfigure}{r}{0.45\textwidth}
		\centering
		\includegraphics[width=0.44\textwidth]{Enddokumentation/Loesungskonzept/Bilder/Blockschaltbild_Controller.png}
		\caption{Blockschaltbild der Controller-Hardware}
		\label{fig:Blockschaltbild_Controller}
	\end{wrapfigure}
	Die Controller-Hardware steuert die Motoren der Ballzuführung, der Stepper für die Ausrichtung 
	der Abwurfeinheit und die Motoren zur Beschleunigung der Bälle. Sobald der Controller das Startsignal vom Smartphone erhält, 
	wird dieser den Motor der Schwungräder aktivieren und auf Nenndrehzahl drehen lassen. Weiter erhält der Controller vom Master die Angabe, 
	in welchem Winkel sich der Korb zum Gerät befindet. Anhand dieser werden die benötigten Schritte berechnet und die resultierenden Befehle 
	an die Motorsteuerung werden abgesetzt. Sobald die Abwurfeinheit die richtige Position eingenommen hat, wird die Ballzuführung aktiviert. 
	Mit einem Sensor werden die Bälle gezählt und sobald der letzte Ball abgefeuert wurde, wird dies dem Master wiederum signalisiert zur Ausgabe des Endsignals.\\
	\\	
	Die Abbildung \ref{fig:Blockschaltbild_Controller} zeigt auf, wie die Controller-Hardware aufgebaut wird. 
	Die Brushless-Motor- und Stepper-Ansteuerung wird auf separaten Boards realisiert, wobei die Stepper-Hardware durch die PREN-ET-Gruppe
	entwickelt und in dieser Gruppe eingesetzt wird. Als Schnittstelle zwischen den Boards und dem Controller wird SPI\footnote{\textbf{S}erial \textbf{P}eripheral \textbf{I}nterface, Dabei handelt es sich um ein synchronen seriellen Datenbus} eingesetzt, 
	da ein Hauptchip der Stepper-Ansteuerung nur über SPI angesprochen werden kann. Die Kommunikation mit dem Smartphone wir über UART\footnote{\textbf{U}niversal \textbf{A}synchronous \textbf{R}eceiver \textbf{T}ransmitter, Dabei handelt es sich um ein asynchrone serielle Schnittstelle} stattfinden, 
	das über den FTDI-Chip auf USB emuliert wird. Die Ansteuerung des DC-Motors wird mittels PWM\footnote{\textbf{P}ulse \textbf{W}idth \textbf{M}odulation, Pulsweitenmodulation} realisiert.