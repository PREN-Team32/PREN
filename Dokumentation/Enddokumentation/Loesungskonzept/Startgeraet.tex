\subsubsection{Startgerät}
Die Übermittlung des Startsignals wird mittels Bluetooth realisiert. 
Sender des Signals ist ein Windows Computer mit integriertem Bluetooth 
Adapter (Notebook). Der Empfänger wird ein Android Smartphone sein.
Wie aus der Technologierecherche (siehe Anhangsdokument) zu entnehmen, 
gewährleistet diese Konstellation grosse Freiheit bei der Umsetzung, 
da vor allem die Entwicklung für Android in Java am leichtesten geht 
und im Team insgesamt der grösste Erfahrungspool in dieser Sprache 
vorhanden ist. Zusätzlich verfügen beide Komponenten über einen WLAN-
Adapter auf welchen zurückgeriffen werden könnte, sofern es zu Problemen 
mit dem Aufbau  und der Aufrechterhaltung der Bluetooth Verbindung kommen 
sollte. Ein weiterer Vorteil des Smartphones ist, dass ein benötigtes 
Endsignal über die integrierten Lautsprecher  als akustisches Signal 
oder über den Display als eine visuelle Ausgabe erfolgen kann.
