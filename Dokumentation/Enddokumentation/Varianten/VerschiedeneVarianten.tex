\section{Von der Recherche zum Feinkonzept}
Um Eckpfeiler für die zu Beginn des Projekts benötigten Recherchen zu erhalten, musste das
Problem grob in seine Teilprobleme zerlegt werden. Aus dieser Zerlegung resultierten die
Bereiche Kommunikation zwischen elektronischen Geräten, Möglichkeiten zur Objekterkennung und
Objektverfolgung, diverse Flugobjekte und Fahrantriebe, Bedarf von Dreh- und Wurfmechanismen
sowie ein Versorgungskonzept. \\
Um Übersicht über die Teilprobleme zu behalten wurde eine Funktionsskizze geschaffen, welche
in  Abbildung \ref{fig:Funktionsskizze} ersichtlich ist.\\
\begin{figure}[h!]
	\centering
	\includegraphics[scale=0.73,clip,trim= 14mm 17.8cm 12mm 50mm]
	{Enddokumentation/Varianten/Bilder/Funktionsskizze.pdf}
	\caption{Funktionsskizze}
	\label{fig:Funktionsskizze}
\end{figure}
\\
Nach der Ermittlung dieser Teilprobleme mussten für die einzelnen Bereiche nach Lösungsansätzen
recherchiert werden. Die Resultate dieser Recherche, sowie die danach folgende Bewertung der
gefundenen Lösungen ist aus Platzgründen im Anhangsdokument zu finden. Um die Ergebnisse der Bewertung
sinnvoll als Entscheidungshilfe einsetzen zu können, wurden sie kompakt zu einem Grobkonzept
zusammengefasst. (Siehe Anhangsdokument)\\
\\
Jedes Teilproblem ist bezüglich den Vor- und Nachteilen nach den definierten Zielsetzungen
bewertet worden. Dadurch lassen sich grafisch geeignete Kombinationsmöglichkeiten herleiten und
untereinander vergleichen. Um möglichst vielen differenzierten Ansätzen Rechnung zu tragen,
wurden vier Varianten während einer Diskussionsrunde festgelegt.
\newpage
\begin{landscape}
	\begin{figure}[t]
		\centering
		\includegraphics[scale=0.8,clip,trim= 15mm 37mm 21mm 19mm]
		{Enddokumentation/Varianten/Bilder/Grobkonzept.pdf}
		\caption{Lösungsansätze für die einzelnen Teilprobleme mit den vier gewählten Varianten}
		\label{fig:Grobkonzept}
	\end{figure}
\end{landscape}
\noindent Die blaue Variante ist die Kombination mit den jeweils bestbewerteten Teillösungen (höchste Prozentzahl).
 Die rote Variante basiert auf der Idee, die Bälle in eine Kugel einzuschliessen,
das Gerät parallel zur Spielfeldwand zu verschieben und den Korb mit einer Smartphone-Kamera zu
erkennen. Der Ballwerfer soll durch einen Akkumulator mit Energie versorgt werden. Als
Ausgangslage führt die grüne Variante die Bälle in einem Drehkranz und befördert diese einzeln
in den Korb, die restlichen Kriterien werden kongruent zur zweiten Variante ausgeführt. Im
orangen Konzept befördert der Ballwerfer die Bälle aus der Startposition in bogenförmiger Kurve in den Korb. Die Ausgabe der Bälle erfolgt vereinzelt. Die übrigen Teilprobleme verwenden
wiederum, äquivalent zur zweiten Variante eine Smartphone-Kamera zur Korberkennung und einen
Akkumulator als Energieversorgung.\\
\\
Die Entscheidung fiel auf die orange Variante. Diese bietet als gesamtes Konzept die
erfolgversprechendste und effizienteste Lösung, bezüglich der Zielsetzung des Teams. \\

\begin{tabular}{p{1cm}p{10cm}}
	\multirow{3}{4cm}	{\includegraphics[width=1cm]{Enddokumentation/Varianten/Bilder/info_icon.png}}
	 & Die detaillierte Beschreibung der Lösungsfindung (von der Funktionsskizze bis zum
	 Feinkonzept) war Aufgabe des zweiten Testates und ist als Dokument im Anhangsdokument beigelegt. \\
\end{tabular}\\
\\
\\
Nach der Entscheidung für eine Variante folgt die weitere Ausarbeitung des Konzeptes
in einzelne Feinkonzepte. Die ursprünglich sieben Teilprobleme wurden in 19 Subteilprobleme
aufgesplittet. Zu jedem Subteilproblem existieren wiederum Lösungsvarianten. Im Unterschied 
zum Grobkonzept erfolgt die Bewertung nicht mit Prozenten, sondern werden die Lösungsvarianten
miteinander verglichen und nach aktuellem Wissenstand eine oder eventuell auch mehrere
Lösungsvarianten ausgewählt. 


