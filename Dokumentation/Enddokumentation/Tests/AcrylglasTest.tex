\subsection{Acrylglas}
\begin{tabular}{p{3.6cm}p{9.4cm}}
	\textit{Typ}              & Lagerung und Bohrung in Acrylglas  \\ 
	\textit{Datum}:           & 21.11.2014   \\
	\textit{Ort}:             & Labor HSLU \\
	\textit{Tester}:          & Matteo, Pascal\\
	\textit{Ziel des Testes}: & Lassen sich Wälzlager in Acrylglas (PMMA) einpressen und halten sie den herrschenden Druck stand? Verhalten, Möglichkeiten von Bohrungen in 5mm Acrylglas.   \\
	\textit{Fazit / Verbesserungs-\newline vorschlag}: & Beim ersten Versuch sind Spannungsrisse aufgetreten (siehe Pfeil in Abbildung \ref{abb:LagerPlexiglas}). 
	Um dem entgegenzuwirken, wurde in einem zweiten Versuch die 16mm Bohrung mit 
	Schleifpapier gering vergrössert, damit sich das Lager leichter einpressen lässt. 
	Die Kräfte, die durch das Lager aufgenommen werden können, sind nun zwar 
	geringer, allerdings für den geplanten Einsatzbereich immer noch genügend. 
	Durch diese Methode treten auch keine Spannungsrisse mehr auf.  \\
	
	Die Bohrung sollte mit einem sehr scharfen Bohrer mit stumpfem Winkel 
	gemacht werden. Weiter sollte sie gekühlt werden, um ein Durchschmelzen 
	durch die sehr dünne, noch verbleibende Restwandstärke, zu verhindern.  \\
	
	\textit{Ziel erreicht}:& Ja\\
\end{tabular}
\begin{figure}[h!]
	\includegraphics[width=5cm]{Funktionstests/Bilder/LagerPlexiglas.jpg}
	\centering
	\caption{Spannungsrisse in Acrylglas-Funktionsmuster} 
	\label{abb:LagerPlexiglas}
\end{figure}