\subsection{Brushless-Motor}
	Dieser Test wurde in der PREN-ET-Gruppe durchgeführt. Dabei handelt es sich um 
	einen komplexeren und grösserern Aufbau.\\

\begin{tabular}{p{3.6cm}p{9.4cm}}
\rule{0pt}{11pt}\textit{Typ}              & Brushless-Motor Ansteuerung \\ 
\rule{0pt}{11pt}\textit{Datum}:           & 6.12.2014   \\
\rule{0pt}{11pt}\textit{Ort}:             & Labor HSLU \\
\rule{0pt}{11pt}\textit{Tester}:          & Yves Studer \& Daniel Winz \\
\rule{0pt}{11pt}\textit{Ziel des Testes}: & Überprüfen, ob der neue Ansatz in der 
Praxis funktioniert und umsetzbar ist\\
\rule{0pt}{11pt}\textit{Fazit / Verbesserungs-\newline vorschlag}: & Die Brushless-
Steuerung funktioniert einwandfrei auch bei einer sehr hohen Drehzahl von bis zu 
$20000\frac{1}{min}$. Mittels einer Wirbelstrombremse konnte $120 W$ elektrische 
Leistung umgesetzt werden. Es zeigte sich, dass eine Regelung benötigt wird, um 
eine konstante Drehzahl auch unter Last zu erreichen.
\end{tabular}
    \input{\EtPath/Testbericht}