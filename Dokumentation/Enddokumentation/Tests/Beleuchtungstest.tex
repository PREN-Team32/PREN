\subsection{Beleuchtungstest}
\begin{tabular}{p{3.6cm}p{9.4cm}}
\rule{0pt}{11pt}\textit{Typ}              & 4 Bauscheinwerfer \\ 
\rule{0pt}{11pt}\textit{Datum}:           & 04.12.2014   \\
\rule{0pt}{11pt}\textit{Ort}:             & originales Spielfeld \\
\rule{0pt}{11pt}\textit{Tester}:          & Gruppe 32 \\
\rule{0pt}{11pt}\textit{Ziel des Testes}: & Den Schattenwurf, welcher von der Beleuchtung verursacht wird auf der Stützwand hinter dem Kübel zu rekonstruieren. \\
\rule{0pt}{11pt}\textit{Fazit / Verbesserungs-\newline vorschlag}: & 
Der Schattenwurf kann vernachlässigt werden. Die Bilderkennung wird durch den Schatten nicht gestört. 
\end{tabular}

\label{chap:LichtTest}\begin{tabular}{p{3.6cm}p{9.4cm}}
	\rule{0pt}{11pt}\textit{Typ}              & Einfluss der Lichtverhältnisse auf das Resultat der Korberkennung  \\ 
	\rule{0pt}{11pt}\textit{Datum}:           & 04.12.2014   \\
	\rule{0pt}{11pt}\textit{Ort}:             & Teaminseln, Trakt 3 HSLU \\
	\rule{0pt}{11pt}\textit{Tester}:          & Thomas, Pascal, Livio, Matteo, Niklaus\\
	\rule{0pt}{11pt}\textit{Ziel des Testes}: & Es soll festgestellt werden, ob der entwickelte Algorithmus zur Korberkennung bei Verwendung von starken Scheinwerfern (und bei dem daraus resultierenden Schattenwurf) funktioniert.  \\
	\rule{0pt}{11pt}\textit{Fazit / Verbesserungs-\newline vorschlag}: & Das Modellspielfeld wurde mit zwei ??? Watt Scheinwerfern ausgeleuchtet und anschliessend mit mehreren Smartphones (LG Nexus, HTC One und Nokia Lumia) abfotografiert. Die Resultate wurden dem Korberkennungs-Algorithmus übergeben, welcher in allen die Position des Korbes bestimmen konnte.  \\
\end{tabular}