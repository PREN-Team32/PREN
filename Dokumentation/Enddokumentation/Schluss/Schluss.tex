\section{Schlussdiskussion}
In einer ersten Phase wurde die Aufgabenstellung in Teilprobleme zerlegt. Für diese partiellen 
Probleme konnte anschliessend nach bestehenden Lösungen recherchiert werden, welche wiederum 
eine Bewertung erhielten und in unterschiedlichen Konstellationen mehrere Grobkonzepte bildeten.\\
\\
Bei der Schaffung der Grobkonzepte mussten einige Grundprobleme angegangen werden. So wurde 
festgelegt, dass ein statischer Werfer konstruiert wird, welcher sich nicht von der Startposition 
wegbewegt. Dies, da sich bei der Verwendung eines Fahrwerks lediglich neue Problemfelder wie 
Rückstoss oder Bestimmung der Eigenposition auftun. Die Verwendung eines Smartphones bot sich 
an, weil dadurch Drahtlos-Adapter, Rechenkapazität und Kamera in einer einzigen Komponente zur Verfügung 
stehen. \\
\\
Dieses grobe erste Konzept wurde weiter ausgearbeitet und daraus ein Feinkonzept geschaffen, 
welches die Grundlage für die Realisierung des Projekts ist: Die Lokalisierung des Korbes wird 
mit einer eigens entwickelten Smartphone-App realisiert. Zum Abwurf der Bälle sollen konkave 
Schwungräder dienen, wobei die Zuführung der Bälle über ein Förderband 
geschieht. \\
\\
Die Elektrotechnik Studierenden haben sich zu einer Gruppe zusammengeschlossen, um 
teamübergreifend gemeinsame Probleme zu lösen. Dies hat sich dahingehend ausgezahlt, dass 
der Brushless Ansteuerungstest mit dem FPGA mit weniger Aufwand als mit einer diskreten Schaltung 
aufgebaut werden konnte. Weiter gibt es Probleme, wie die Stepper-Ansteuerung, die von einer anderen 
Gruppe erarbeitet wird. Ein weiterer wichtiger 
Aspekt dieser Zusammenarbeit ist, dass realitätsnah eine übergeordnete Kollaboration geübt werden 
kann.\\
\\
Insgesamt wurde somit ein Konzept geschaffen, welches die gesetzten Ziele und 
Rahmenbedingungen erfüllt und den sofortigen Beginn der Umsetzung Anfang nächstes Semester 
ermöglicht.