\subsection{Ausblick}
Im nächsten Semester wird das erarbeitete Konzept realisiert. Aus administrativer 
Sicht muss vermehrt und besser zu Beginn einer jeden Meilensteinphase abgesprochen 
werden, welche Dokumente mit welchem spezifischen Inhalt erstellt werden müssen. 
Während dem PREN1-Modul musste jedes Mal vor einem Meilenstein ein Grosszahl der 
Dokumente überarbeitet oder gar neu erstellt werden, weil sich zeigte, dass der 
Inhalt nicht dem Geforderten entsprach.\\
\\
Mechanisch gesehen steht das Konzept. Einzelne Teilprobleme wie beispielsweise 
das Zuführen der Bälle zu den Schwungräder mittels Leitblech müssen noch konstruiert 
werden. Das Gewicht der Abwurfeinheit soll zudem weiter optimiert werden, um das 
Gewichtslimit einzuhalten. Um die einzelnen Bauteile fertigen zu können, müssen 
zunächst die CAD Zeichnungen erstellt werden. Des Weiteren muss die Software 
entwickelt werden. Es wurden bereits Tests zur Bluetooth-Kommunikation, der 
Objekterkennung und der Kommunikation zwischen Controller und Smartphone 
durchgeführt. Diese einzelnen Komponenten müssen jedoch in einer homogenen 
Smartphone-App integriert werden. Zu Test- und Demonstrationszwecken soll ausserdem 
der DesktopViewer weiter ausgebaut werden, um Feinabstimmungen über eine 
GUI-Applikation zu ermöglichen.\\
\\
Ungewissheit besteht ebenfalls betreffend der Mitarbeit von Manuel Güdel. Da er 
das Modul PREN 2 bereits letztes Jahr absolviert hat, ist noch nicht sicher, ob 
er nächstes Semester offiziell an der Entwicklung weiterarbeiten kann. Inoffiziell 
hat er bereits seine Hilfe angeboten und wird das Team weiterhin so gut wie möglich 
unterstützen.
