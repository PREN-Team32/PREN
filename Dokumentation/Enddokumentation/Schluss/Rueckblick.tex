\subsection{Rückblick}
Eine wichtige Erkenntnis wurde im PREN 1 Modul bezüglich kollaboratives Arbeiten 
gewonnen. Der Austausch aller Dateien fand via Dropbox statt. Dieses Modell zeigte 
jedoch mit der Zeit Schwächen, vor allem bezüglich des gemeinsamen editierens eines 
Dokuments. Es wäre daher eine Diskussion wert, die Dokumente welche von mehreren 
Teammitgliedern bearbeitet werden auf eine Plattform für kollaboratives Arbeiten 
(z.B. Google Drive) auszulagern und die Dropbox als reinen Dateienaustausch 
beizubehalten.\\
\\
Wir haben bewusst eine flache Teamhierarchie gewählt und managen unser Projekt 
so, dass jedes Teammitglied die Verantwortung für einen festgelegten Bereich 
erhält. Generelle Diskussionen zu Projektmanagement-Themen werden im Plenum 
besprochen und allfällige Entscheide gegebenenfalls gefällt. Bis zum jetzigen 
Zeitpunkt funktioniert diese Art der Zusammenarbeit ohne grössere Probleme und 
Reibereien. Es hat sich gezeigt, dass man in Diskussionen rasch vom eigentlichen 
Thema abweicht oder nicht mehr zielgerichtet diskutiert. Alle Teammitglieder 
sollten vermehrt darauf bedacht sein, themenbezogen und effizient an einer Lösung 
zu diskutieren.
