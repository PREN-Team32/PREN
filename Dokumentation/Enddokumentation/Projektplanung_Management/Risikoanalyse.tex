\subsection{Risikoanalyse}
Für ein erfolgreiches Gelingen des PREN-Projekts ist es notwendig, vorgängig die 
Risiken des Produkts abzuschätzen, um mögliche Probleme frühzeitig zu erkennen 
und zu umgehen. Erkennbare Risiken wurden im Team initialisiert, in einer Tabelle 
aufgelistet und bewertet. Die nachfolgenden Risiken werden als erheblich 
(Farbcode rot) eingestuft und müssen für die Realisierung des Produkts beachtet 
werden. Die komplette Risikoliste ist im Anhangsdokument zu finden.
%
 \begin{longtable}{p{4cm}l l p{5.5cm}}
 	\textbf{Risiko} & \textbf{Auswirkung} & \textbf{Wahrschein-} & \textbf{Massnahmen} \\
 	                &                     & \textbf{lichkeit}    & \\	     
 	\hline & & & \\
 	\endhead
	\rule{0pt}{12pt}Zeitdruck lässt Realisierung des ursprüngliches Konzept nicht mehr zu &
	Kritisch & Wahrscheinlich & Auf wichtigste Elemente des Prototyps konzentrieren und eine, 
	falls möglich, vereinfachte Version bauen. \\
	
	\rule{0pt}{12pt}Wurfgenauigkeit entspricht nicht den Mindestanforderungen 
	(innerhalb $20 cm$ x $20 cm$) & Katastrophal & Wahrscheinlich & Sofort Teamsitzungen 
	einberufen, um Massnahmen und Lösungen zu suchen/recherchieren evtl. unter Einbezug der 
	Dozenten.\\
	
	\rule{0pt}{12pt}Zu hohe Abbremsung der Schwungräder, wenn die Bälle beschleunigt werden & 
	Kritisch & Wahrscheinlich & Dimensionierung und Ansteuerung der Motoren entsprechend 
	definieren.\\
	
	\rule{0pt}{12pt}Das Gesamtgewicht der Konstruktion ist grösser als $2 kg$. & Kritisch & 
	Wahrscheinlich & Konstruktion und Wahl der einzelnen Teile explizit nach geringst 
	möglichem Gewicht wählen (gilt während des ganzen Projekts)\\
	
	\rule{0pt}{12pt}Korberkennung mit Android Device nicht möglich. & Kritisch & Wahrscheinlich &
	 Korberkennung frühzeitig (Mitte PREN1) als eigenstehendes Modul erstellen und 
	 verifizieren-Schlechte Lichtverhältnisse respektive Schattenwurf (durch Baustrahler) 
	 miteinbeziehen.\\
 	\caption{Risikoanalyse}
 	\label{tab:RikisoanalyseTabelle}
 \end{longtable}
