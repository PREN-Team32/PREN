\subsubsection{Berechnung Horizontale Ausrichtung}
Für die Stelleinheit wird ein Schrittmotor eingesetzt. Der Antriebsstrang muss 
hinsichtlich Drehmoments ausgelegt werden. Die Antriebsritzel wurden so gewählt, 
dass ein möglichst leichter Schrittmotor verwendet werden kann. 
\begin{align}
    F_{Losteil} &= m_{Losteil} \cdot g\\
    F_2 &= F_{Losteil} \cdot 0.1
\end{align}
\begin{align}
    M_{Drehung} &= L \cdot F_{2R}\\
    F_{2R} &= \mu_H \cdot F_2
\end{align}
\begin{align}
    M_{Motor} &= F_z \cdot r\\
    F_z &= \frac{M_{Drehung}}{R}\\
    M_{Motor} &=\frac{L \cdot \mu_H \cdot m_{Losteil} \cdot g \cdot 0.1}{R} \cdot r
\end{align}

\textbf{Berechnungswerte}\\
\begin{tabular}{lll}
	\rule{0pt}{11pt} $m_{Losteil}$ & $1.6 kg$ & \\
	\rule{0pt}{11pt} $g$ & $9.81 \frac{N}{kg}$ & \\
	\rule{0pt}{11pt} $L$ & $0.437 m$ &  \\
	\rule{0pt}{11pt} $\mu_H$ & $0.2$ & Haftreibungskoeffizient zwischen Rolle und Boden \\
	\rule{0pt}{11pt} $r$ & $0.01 m$ & \\
	\rule{0pt}{11pt} $R$ & $0.28 m$ & \\
\end{tabular}\\
\\
\\
\textbf{Resultate}\\
\begin{tabular}{ll}
	\rule{0pt}{11pt} $M_{motor}$ & $0.0049Nm$ \\
\end{tabular}