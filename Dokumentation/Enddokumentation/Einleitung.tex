\section{Einleitung}
Im Rahmen des Produktentwicklungsmoduls (PREN) erhalten interdisziplinäre Teams die Aufgabe, einen autonomen Ballwerfer zu erarbeiten. Das Ziel besteht darin, möglichst viele der fünf Tennisbälle, in möglichst kurzer Zeit, in einen Korb zu befördern. Als weiteres Bewertungskriterium gilt das Gewicht des Produkts, welches ab zwei Kilogramm ein stufenweiser Punkteabzug zur Folge hat. Der Korb befindet sich in einem Spielfeld - welches seitlich und in der Höhe begrenzt ist - am hinteren Ende an einer Wand und ist horizontal verschiebbar. Die endgültige Position des Korbes wird kurz vor der Abgabe des Startsignals durch einen Dozent festgelegt. Die Übermittlung des Startsignals muss drahtlos erfolgen, nach Ausführen der Aufgabe, muss das Endsignal akustisch erfolgen.\\
\\
Ein interdisziplinäres Team besteht aus Studenten der drei Disziplinen Elektrotechnik, Maschinenbau und Informatik.\\
\\
Das Produktentwicklungsmodul ist in zwei Teile aufgeteilt, das PREN1-Modul im Herbstsemester sowie das PREN2-Modul im Frühlingssemester. Wichtigste Aufgabe im PREN1-Modul ist das erarbeiten eines Konzepts, eine professionelle, strukturierte Projektabwicklung und das Verifizieren kritischer Teilprobleme mittels Funktionsmuster. Die Realisation des erarbeiteten Konzepts wird im PREN2-Modul in Angriff genommen. 

